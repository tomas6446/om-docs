\documentclass[a4paper,12pt,fleqn]{article}
\input{allPacks}

\newtoggle{inLithuanian}
\settoggle{inLithuanian}{true}

\usepackage{minted}
\usemintedstyle{emacs}
\usepackage{amsmath}
\usepackage{spverbatim}

\usepackage{booktabs}

\usepackage{fontspec}
\setmainfont{Times New Roman}
\usepackage{polyglossia}
\setmainlanguage{lithuanian}

\usepackage{listings}
\usepackage{xcolor}

\newenvironment{conditions}
  {\par\vspace{\abovedisplayskip}\noindent\begin{tabular}{>{$}l<{$} @{${}={}$} l}}
  {\end{tabular}\par\vspace{\belowdisplayskip}}

\newtoggle{needPreface}
\settoggle{needPreface}{false}

\newtoggle{signaturesOnTitlePage}
\settoggle{signaturesOnTitlePage}{false}

\graphicspath{ {./images/} }

\input{macros}

\usepackage[german=quotes]{csquotes}
\DeclareQuoteAlias{german}{lithuanian}

\begin{document}
 \depttitlepage{Optimizavimo metodai. Informatika. 3 kursas.}
 {Tiesinis programavimas}
 {Tomas Kozakas} 
 {}{}{}{}
 {doc. dr. Arvydas Kregždė}

\tableofcontents
\pagebreak

\section{Įvadas}
\subsection{Uždavinio aprašymas}
Šiame darbe nagrinėjama tiesinio programavimo problema, kuriai spręsti naudojamas simplekso algoritmas. Simplekso algoritmas yra matematinis optimizavimo metodas, naudojamas tiesinio programavimo problemoms spręsti.
Uždavinys formuluojamas taip:

Tikslo funkcija, kurią reikia minimizuoti:
\begin{align}
2x_1 - 3x_2 - 5x_4
\end{align}

Sprendimas turi atitikti šiuos apribojimus:
\begin{equation}
\begin{aligned}
-x_1 + x_2 - x_3 - x_4 &\leq 8, \\
2x_1 + 4x_2 &\leq 10, \\
x_3 + x_4 &\leq 3, \\
x_i &\geq 0.
\end{aligned}
\end{equation}

Pirmiausia tiesinio programavimo problemą perrašom matriciniu pavidalu. Pradinės konstantos 8, 10, 3 yra apribojimų konstantos ir konstantos a, b ir c yra studento knygelės numerio \enquote{1x1xabc} skaitmenys individualam uždaviniui.

Tikslo funkcija, kurią reikia minimizuoti:
\begin{align}
2x_1 - 3x_2 - 5x_4
\end{align}

Asmeninis sprendimas turi atitikti šiuos apribojimus:
\begin{equation}
\begin{aligned}
-x_1 + x_2 - x_3 - x_4 &\leq 3, \\
2x_1 + 4x_2 &\leq 0, \\
x_3 + x_4 &\leq 6, \\
x_i &\geq 0.
\end{aligned}
\end{equation}

\pagebreak
\section{Uždavinio suvedimas į standartinę formą}

Pirmasis žingsnis sprendžiant optimizavimo uždavinį yra suvesti jį į standartinę formą. 
Standartinėje formoje uždavinį galima užrašyti matriciniu pavidalu, kaip parodyta \ref{eq:matrix} lygtyje. Čia $C$ yra tikslinės funkcijos koeficientų matrica, $A$ yra apribojimų koeficientų matrica, $B$ yra apribojimų vertės, o $X$ yra kintamųjų matrica, kurią siekiame optimizuoti.

\begin{equation} \label{eq:matrix}
\begin{aligned}
\min \quad & CX, \\
\quad & AX = B, \\
& X \geq 0.
\end{aligned}
\end{equation}

Matricos:

\begin{equation}
C = \begin{pmatrix}
2 & -3 & 0 & -5 & 0 & 0 & 0
\end{pmatrix}
\end{equation}

\begin{equation}
A = \begin{pmatrix}
-1 & 1 & -1 & 1 & 1 & 0 & 0 \\
2 & 4 & 0 & 0 & 0 & 1 & 0 \\
0 & 0 & 1 & 1 & 0 & 0 & 1
\end{pmatrix}
\end{equation}

\begin{equation}
B = \begin{pmatrix}
8 & 10 & 3
\end{pmatrix}
\end{equation}

Atvaizduotos visą informaciją \ref{tab:first-table} lentelėje:

\begin{table}[h!]
\centering
\caption{Suformuota lentelė.}
\label{tab:first-table}
\begin{tabular}{@{}lccccccc@{}}
\toprule
0 & 2 & -3 & 0 & -5 & 0 & 0 & 0 \\ \midrule
8 & -1 & 1 & -1 & -1 & 1 & 0 & 0 \\
10 & 2 & 4 & 0 & 0 & 0 & 1 & 0 \\
3 & 0 & 0 & 1 & 1 & 0 & 0 & 1 \\ \bottomrule
\end{tabular}
\end{table}

\pagebreak
\section{Uždavinio sprendimas}
\subsection{Pagrindinio uždavinio sprendimas}

Sprendinys buvo rastas per dvi iteracijas. Pradinės reikšmės, reikšmės po pirmosios ir antrosios iteracijos pateikiamos lentelėse \ref{tab:generic_task_table}, \ref{tab:generic_task_table1} ir \ref{tab:generic_task_table2} atitinkamai. Po dviejų iteracijų algoritmas pateikė šiuos rezultatus:

\begin{enumerate}
\item Minimumo taškas: $\left(0, 2.5, 0, 3.0\right)$
\item Minimali tikslo funkcijos reikšmė: $-22.5$
\item Bazė: $\{2, 4, 5\}$
\end{enumerate}

\begin{table}[h!]
\centering
\caption{Pradinė uždavinio lentelė.}
\label{tab:generic_task_table}
\begin{tabular}{@{}lccccccc@{}}
\toprule
0 & 2 & -3 & 0 & -5 & 0 & 0 & 0 \\ \midrule
8 & -1 & 1 & -1 & -1 & 1 & 0 & 0 \\
10 & 2 & 4 & 0 & 0 & 0 & 1 & 0 \\
3 & 0 & 0 & 1 & 1 & 0 & 0 & 1 \\ \bottomrule
\end{tabular}
\end{table}

\begin{table}[h!]
\centering
\caption{Pagrindinio uždavinio lentelė po pirmosios iteracijos.}
\label{tab:generic_task_table1}
\begin{tabular}{@{}lcccccccc@{}}
\toprule
15 & 2.0 & -3.0 & 5.0 & 0.0 & 0.0 & 0.0 & 5.0 \\ \midrule
11 & -1.0 & 1.0 & 0.0 & 0.0 & 1.0 & 0.0 & 1.0 \\
10 & 2.0 & 4.0 & 0.0 & 0.0 & 0.0 & 1.0 & 0.0 \\
3 & 0.0 & 0.0 & 1.0 & 1.0 & 0.0 & 0.0 & 1.0 \\ \bottomrule
\end{tabular}
\end{table}

\begin{table}[h!]
\centering
\caption{Pagrindinio uždavinio lentelė po antrosios iteracijos.}
\label{tab:generic_task_table2}
\begin{tabular}{@{}lccccccc@{}}
\toprule
22.5 & 3.5 & 0.0 & 5.0 & 0.0 & 0.0 & 0.75 & 5.0 \\ \midrule
8.5 & -1.5 & 0.0 & 0.0 & 0.0 & 1.0 & -0.25 & 1.0 \\
2.5 & 0.5 & 1.0 & 0.0 & 0.0 & 0.0 & 0.25 & 0.0 \\
3.0 & 0.0 & 0.0 & 1.0 & 1.0 & 0.0 & 0.0 & 1.0 \\ \bottomrule
\end{tabular}
\end{table}

\pagebreak
\subsection{Asmeninio uždavinio sprendimas}

Algoritmas rado optimalų uždavinio sprendinį per dvi iteracijas. Pradinės reikšmės bei reikšmės po pirmosios ir antrosios iteracijos atvaizduotos lentelėse \ref{tab:personal_task_table}, \ref{tab:personal_task_table1} ir \ref{tab:personal_task_table2} atitinkamai. Pasibaigus iteracijoms, algoritmas pateikė tokius rezultatus:

\begin{enumerate}
    \item Minimumo taškas: $\left(0, 0, 0, 6.0\right)$
    \item Minimali tikslo funkcijos reikšmė: $-30.0$
    \item Bazė: $\{2, 4, 5\}$
\end{enumerate}

\begin{table}[h!]
\centering
\caption{Asmeninio uždavinio lentelė.}
\label{tab:personal_task_table}
\begin{tabular}{@{}lccccccc@{}}
\toprule
0 & 2 & -3 & 0 & -5 & 0 & 0 & 0 \\ \midrule
3 & -1 & 1 & -1 & -1 & 1 & 0 & 0 \\
0 & 2 & 4 & 0 & 0 & 0 & 1 & 0 \\
6 & 0 & 0 & 1 & 1 & 0 & 0 & 1 \\ \bottomrule
\end{tabular}
\end{table}

\begin{table}[h!]
\centering
\caption{Asmeninio uždavinio lentelė po pirmos iteracijos.}
\label{tab:personal_task_table1}
\begin{tabular}{@{}lccccccc@{}}
\toprule
30.0 & 2.0 & -3.0 & 5.0 & 0.0 & 0.0 & 0.0 & 5.0 \\ \midrule
9.0 & -1.0 & 1.0 & 0.0 & 0.0 & 1.0 & 0.0 & 1.0 \\
0.0 & 2.0 & 4.0 & 0.0 & 0.0 & 0.0 & 1.0 & 0.0 \\
6.0 & 0.0 & 0.0 & 1.0 & 1.0 & 0.0 & 0.0 & 1.0 \\ \bottomrule
\end{tabular}
\end{table}

\begin{table}[h!]
\centering
\caption{Asmeninio uždavinio lentelė po antros iteracijos.}
\label{tab:personal_task_table2}
\begin{tabular}{@{}lccccccc@{}}
\toprule
30.0 & 3.5 & 0.0 & 5.0 & 0.0 & 0.0 & 0.75 & 5.0 \\ \midrule
9.0 & -1.5 & 0.0 & 0.0 & 0.0 & 1.0 & -0.25 & 1.0 \\
0.0 & 0.5 & 1.0 & 0.0 & 0.0 & 0.0 & 0.25 & 0.0 \\
6.0 & 0.0 & 0.0 & 1.0 & 1.0 & 0.0 & 0.0 & 1.0 \\ \bottomrule
\end{tabular}
\end{table}

\pagebreak
\section{Palyginimas ir išvados}

Lentelėje yra suvesti algoritmo rezultatai (žr. \ref{tab:results} lent.), gauti sprendžiant du skirtingus uždavinius.
Nors gauti minimumo taškai ir minimalios tikslo funkcijos reikšmės skiriasi, abu uždaviniai pasižymi ta pačia baze - $\{2, 4, 5\}$.

Apibendrinus, galime teigti, kad suprogramuotas simplekso algoritmas sėkmingai išsprendė tiek pagrindinį, tiek asmeninį uždavinius. Abu uždaviniai buvo išspręsti efektyviai, rasta minimali funkcijų reikšmė, optimalūs sprendimai ir atitinkamos bazės.

\begin{table}[h!]
\centering
\caption{Algoritmo rezultatai}
\begin{tabular}{@{}lccc@{}}
\toprule
Uždavinys & Minimumo taškas & Minimali tikslo funkcijos reikšmė & Bazė \\ \midrule
Pagrindinis & $\left(0, 2.5, 0, 3.0\right)$ & $-22.5$ & $\{2, 4, 5\}$ \\
Asmeninis & $\left(0, 0, 0, 6.0\right)$ & $-30.0$ & $\{2, 4, 5\}$ \\ \bottomrule
\end{tabular}
\label{tab:results}
\end{table}

\pagebreak
\section{Priedai}
\subsection{Programos išvestis} \label{output}

\begin{minted}[breaklines, fontfamily=tt, autogobble]{text}
=====================================
Optimizing the generic task:
=====================================
Current table:
[    2.00 |    -3.00 |     0.00 |    -5.00 |     0.00 |     0.00 |     0.00],     0.00
[   -1.00 |     1.00 |    -1.00 |    -1.00 |     1.00 |     0.00 |     0.00],     8.00
[    2.00 |     4.00 |     0.00 |     0.00 |     0.00 |     1.00 |     0.00],    10.00
[    0.00 |     0.00 |     1.00 |     1.00 |     0.00 |     0.00 |     1.00],     3.00
Smallest objective function coefficient (column index): 3
Smallest non-negative ratio index (row index): 3
Pivot: 1
[2, -3, 0, -5, 0, 0, 0], 0
[-1, 1, -1, -1, 1, 0, 0], 8
[2, 4, 0, 0, 0, 1, 0], 10
[0.0, 0.0, 1.0, 1.0, 0.0, 0.0, 1.0], 3.0
---------
Ratio: -5.0
Ratio: -1.0
Ratio: 0.0
[2.0, -3.0, 5.0, 0.0, 0.0, 0.0, 5.0], 15.0
[-1.0, 1.0, 0.0, 0.0, 1.0, 0.0, 1.0], 11.0
[2.0, 4.0, 0.0, 0.0, 0.0, 1.0, 0.0], 10.0
[0.0, 0.0, 1.0, 1.0, 0.0, 0.0, 1.0], 3.0
=====================================
Waiting for input. Press any key...


=====================================
Current table:
[    2.00 |    -3.00 |     5.00 |     0.00 |     0.00 |     0.00 |     5.00],    15.00
[   -1.00 |     1.00 |     0.00 |     0.00 |     1.00 |     0.00 |     1.00],    11.00
[    2.00 |     4.00 |     0.00 |     0.00 |     0.00 |     1.00 |     0.00],    10.00
[    0.00 |     0.00 |     1.00 |     1.00 |     0.00 |     0.00 |     1.00],     3.00
Smallest objective function coefficient (column index): 1
Smallest non-negative ratio index (row index): 2
Pivot: 4.0
[2.0, -3.0, 5.0, 0.0, 0.0, 0.0, 5.0], 15.0
[-1.0, 1.0, 0.0, 0.0, 1.0, 0.0, 1.0], 11.0
[0.5, 1.0, 0.0, 0.0, 0.0, 0.25, 0.0], 2.5
[0.0, 0.0, 1.0, 1.0, 0.0, 0.0, 1.0], 3.0
---------
Ratio: -3.0
Ratio: 1.0
Ratio: 0.0
[3.5, 0.0, 5.0, 0.0, 0.0, 0.75, 5.0], 22.5
[-1.5, 0.0, 0.0, 0.0, 1.0, -0.25, 1.0], 8.5
[0.5, 1.0, 0.0, 0.0, 0.0, 0.25, 0.0], 2.5
[0.0, 0.0, 1.0, 1.0, 0.0, 0.0, 1.0], 3.0
=====================================
Waiting for input. Press any key...



=====================================
Final Tableau:
[    3.50 |     0.00 |     5.00 |     0.00 |     0.00 |     0.75 |     5.00],    22.50
[   -1.50 |     0.00 |     0.00 |     0.00 |     1.00 |    -0.25 |     1.00],     8.50
[    0.50 |     1.00 |     0.00 |     0.00 |     0.00 |     0.25 |     0.00],     2.50
[    0.00 |     0.00 |     1.00 |     1.00 |     0.00 |     0.00 |     1.00],     3.00
=====================================
Final Variables: 
x1 =   0.0000 | x2 =   2.5000 | x3 =   0.0000 | x4 =   3.0000 | x5 =   8.5000 | x6 =   0.0000 | x7 =   0.0000

Basis Indices: x2, x4, x5
Optimum Value: -22.5000

Optimized Results:
Point:     0.00 |     2.50 |     0.00 |     3.00
Base: x2 | x4 | x5
Optimum:   -22.50
=====================================


=====================================
Optimizing the personal task:
=====================================
Current table:
[    2.00 |    -3.00 |     0.00 |    -5.00 |     0.00 |     0.00 |     0.00],     0.00
[   -1.00 |     1.00 |    -1.00 |    -1.00 |     1.00 |     0.00 |     0.00],     3.00
[    2.00 |     4.00 |     0.00 |     0.00 |     0.00 |     1.00 |     0.00],     0.00
[    0.00 |     0.00 |     1.00 |     1.00 |     0.00 |     0.00 |     1.00],     6.00
Smallest objective function coefficient (column index): 3
Smallest non-negative ratio index (row index): 3
Pivot: 1
[2, -3, 0, -5, 0, 0, 0], 0
[-1, 1, -1, -1, 1, 0, 0], 3
[2, 4, 0, 0, 0, 1, 0], 0
[0.0, 0.0, 1.0, 1.0, 0.0, 0.0, 1.0], 6.0
---------
Ratio: -5.0
Ratio: -1.0
Ratio: 0.0
[2.0, -3.0, 5.0, 0.0, 0.0, 0.0, 5.0], 30.0
[-1.0, 1.0, 0.0, 0.0, 1.0, 0.0, 1.0], 9.0
[2.0, 4.0, 0.0, 0.0, 0.0, 1.0, 0.0], 0.0
[0.0, 0.0, 1.0, 1.0, 0.0, 0.0, 1.0], 6.0
=====================================
Waiting for input. Press any key...

=====================================
Current table:
[    2.00 |    -3.00 |     5.00 |     0.00 |     0.00 |     0.00 |     5.00],    30.00
[   -1.00 |     1.00 |     0.00 |     0.00 |     1.00 |     0.00 |     1.00],     9.00
[    2.00 |     4.00 |     0.00 |     0.00 |     0.00 |     1.00 |     0.00],     0.00
[    0.00 |     0.00 |     1.00 |     1.00 |     0.00 |     0.00 |     1.00],     6.00
Smallest objective function coefficient (column index): 1
Smallest non-negative ratio index (row index): 2
Pivot: 4.0
[2.0, -3.0, 5.0, 0.0, 0.0, 0.0, 5.0], 30.0
[-1.0, 1.0, 0.0, 0.0, 1.0, 0.0, 1.0], 9.0
[0.5, 1.0, 0.0, 0.0, 0.0, 0.25, 0.0], 0.0
[0.0, 0.0, 1.0, 1.0, 0.0, 0.0, 1.0], 6.0
---------
Ratio: -3.0
Ratio: 1.0
Ratio: 0.0
[3.5, 0.0, 5.0, 0.0, 0.0, 0.75, 5.0], 30.0
[-1.5, 0.0, 0.0, 0.0, 1.0, -0.25, 1.0], 9.0
[0.5, 1.0, 0.0, 0.0, 0.0, 0.25, 0.0], 0.0
[0.0, 0.0, 1.0, 1.0, 0.0, 0.0, 1.0], 6.0
=====================================
Waiting for input. Press any key...


=====================================
Final Tableau:
[    3.50 |     0.00 |     5.00 |     0.00 |     0.00 |     0.75 |     5.00],    30.00
[   -1.50 |     0.00 |     0.00 |     0.00 |     1.00 |    -0.25 |     1.00],     9.00
[    0.50 |     1.00 |     0.00 |     0.00 |     0.00 |     0.25 |     0.00],     0.00
[    0.00 |     0.00 |     1.00 |     1.00 |     0.00 |     0.00 |     1.00],     6.00
=====================================
Final Variables: 
x1 =   0.0000 | x2 =   0.0000 | x3 =   0.0000 | x4 =   6.0000 | x5 =   9.0000 | x6 =   0.0000 | x7 =   0.0000

Basis Indices: x2, x4, x5
Optimum Value: -30.0000

Optimized Results:
Point:     0.00 |     0.00 |     0.00 |     6.00
Base: x2 | x4 | x5
Optimum:   -30.00
=====================================
\end{minted}

\pagebreak
\subsection{Programos kodas} \label{source}
\begin{minted}[breaklines, fontfamily=tt, linenos, autogobble]{Python}
import numpy as np

verbose = True


class TableRow:
    def __init__(self, cf: [float], fval: float):
        self.cf = cf
        self.fval = fval

    def __str__(self) -> str:
        return f"{self.cf}, {self.fval}"


def pivoting(index: int, full_table: [TableRow], tolerance=1e-5) -> int:
    min_ratio = float("inf")
    pivot_row_index = -1

    for i, row in enumerate(full_table[1:]):
        if row.cf[index] > tolerance:
            ratio = row.fval / row.cf[index]
            # Bland's Rule: choose the smallest index if there's a tie To avoid cycling, Bland's Rule suggests always
            # choosing the smallest index that satisfies the pivot selection criteria.
            if ratio < min_ratio or (abs(ratio - min_ratio) < tolerance and i < pivot_row_index):
                min_ratio = ratio
                pivot_row_index = i

    if pivot_row_index == -1:
        # Handle the case where no valid pivot is found
        return -1

    return pivot_row_index + 1  # +1 to account for the objective function row


def adjust_table(
        full_table: [TableRow], pivot_row_index: int, pivot_col_index: int
) -> None:
    pivot = full_table[pivot_row_index].cf[pivot_col_index]

    # Divide each element in the pivot row by the pivot.
    # Pivot element should become 1.
    pivot_row = full_table[pivot_row_index]
    pivot_row.fval = pivot_row.fval / pivot
    pivot_row.cf = [x / pivot for x in pivot_row.cf]

    if verbose:
        for row in full_table:
            print(row)
        print("---------")

    for i in range(len(full_table)):
        if i == pivot_row_index:
            continue

        same_col_as_pivot_el = full_table[i].cf[pivot_col_index]
        new_pivot = full_table[pivot_row_index].cf[pivot_col_index]
        ratio = same_col_as_pivot_el / new_pivot
        if verbose:
            print(f"Ratio: {ratio}")

        full_table[i].cf = [
            x - ratio * pivot_row_el
            for x, pivot_row_el in zip(full_table[i].cf, pivot_row.cf)
        ]

        full_table[i].fval = full_table[i].fval - ratio * pivot_row.fval

    if verbose:
        for row in full_table:
            print(row)


def optimize_linear_program(full_table: [TableRow], var_count: int) -> None:
    function_row = full_table[0]

    while not all(num >= 0 for num in function_row.cf):
        if verbose:
            print("=====================================")
            print("Current table:")
            print_table(full_table)

        pivot_col_index = np.argmin(function_row.cf)
        if verbose:
            print(
                f"Smallest objective function coefficient (column index): {pivot_col_index}"
            )

        pivot_row_index = pivoting(pivot_col_index, full_table)
        if verbose:
            print(f"Smallest non-negative ratio index (row index): {pivot_row_index}")

        pivot = full_table[pivot_row_index].cf[pivot_col_index]
        if verbose:
            print(f"Pivot: {pivot}")

        adjust_table(full_table, pivot_row_index, pivot_col_index)

        if verbose:
            print("=====================================")
            input("Waiting for input. Press any key...\n\n")

    column_sums = [
        sum(row.cf[i] for row in full_table) for i in range(len(full_table[0].cf))
    ]
    base_indexes = [index for index, value in enumerate(column_sums) if value == 1]

    final_vars = [0] * var_count

    for row in full_table[1:]:
        for i in base_indexes:
            if row.cf[i] == 1:
                final_vars[i] = row.fval

    if verbose:
        print("=====================================")
        print("Final Tableau:")
        print_table(full_table)

    print("=====================================")
    print("Final Variables: ")
    print(' | '.join(f"x{i + 1} = {val:8.4f}" for i, val in enumerate(final_vars[:var_count])))

    print(f"\nBasis Indices: {', '.join(f'x{index + 1}' for index in base_indexes)}")
    print(f"Optimum Value: {-function_row.fval:8.4f}")

    return final_vars, [x + 1 for x in base_indexes], -function_row.fval


def print_table(full_table):
    for row in full_table:
        formatted_row = ' | '.join(f"{val:8.2f}" for val in row.cf)
        print(f"[{formatted_row}], {row.fval:8.2f}")


def print_results(res: [[float], [float], float], true_var_count: int):
    print("\nOptimized Results:")
    print(f"Point: {' | '.join(f'{val:8.2f}' for val in res[0][:true_var_count])}")
    print(f"Base: {' | '.join(f'x{index}' for index in res[1])}")
    print(f"Optimum: {res[2]:8.2f}")
    print("=====================================")


def main() -> None:
    # We have 7 variables in total. 4 are given and 3 are slack.
    # Left most element of the list is a constant.
    # First row must be an objective function.
    task = [
        TableRow([2, -3, 0, -5, 0, 0, 0], 0),  # <- Objective function
        TableRow([-1, 1, -1, -1, 1, 0, 0], 8),  # <- Constraints
        TableRow([2, 4, 0, 0, 0, 1, 0], 10),
        TableRow([0, 0, 1, 1, 0, 0, 1], 3),
    ]
    print("=====================================")
    print("Optimizing the generic task:")
    print_results(optimize_linear_program(task, 7), 4)

    task_personal = [
        TableRow([2, -3, 0, -5, 0, 0, 0], 0),
        TableRow([-1, 1, -1, -1, 1, 0, 0], 3),
        TableRow([2, 4, 0, 0, 0, 1, 0], 0),
        TableRow([0, 0, 1, 1, 0, 0, 1], 6),
    ]

    print("\n\n=====================================")
    print("Optimizing the personal task:")
    print_results(optimize_linear_program(task_personal, 7), 4)


if __name__ == "__main__":
    main()
\end{minted}

\end{document}
